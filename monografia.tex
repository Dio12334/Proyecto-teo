\documentclass[12pt]{article}
\usepackage[utf8]{inputenc}
\usepackage[spanish]{babel}
\usepackage{listings}
\usepackage{xcolor}

\definecolor{codegreen}{rgb}{0,0.6,0}
\definecolor{codegray}{rgb}{0.5,0.5,0.5}
\definecolor{codepurple}{rgb}{0.58,0,0.82}
\definecolor{backcolour}{rgb}{0.95,0.95,0.92}

\lstdefinestyle{mystyle}{
    backgroundcolor=\color{backcolour},   
    commentstyle=\color{codegreen},
    keywordstyle=\color{magenta},
    numberstyle=\tiny\color{codegray},
    stringstyle=\color{codepurple},
    basicstyle=\ttfamily\footnotesize,
    breakatwhitespace=false,         
    breaklines=true,                 
    captionpos=b,                    
    keepspaces=true,                 
    numbers=left,                    
    numbersep=5pt,                  
    showspaces=false,                
    showstringspaces=false,
    showtabs=false,                  
    tabsize=2
}

\lstset{style=mystyle}

\begin{document}

\title{Monografia}
\author{Diego Paredes}
\maketitle

\tableofcontents

\section{Estructura de la clase AFD}

\lstset{language=C++}
\begin{lstlisting}
class AFD{
  public:
      AFD(int estados, int estado_inicial, std::set<int> estados_finales);
      ~AFD();
      void insertar_transicion(int estado_partida, int estado_llegada, bool transicion);
      void print_afd();
      void brzozowksi();
      void equivalentes();
  private:
      std::vector<std::vector<std::pair<int, bool>>> afd;
      int estado_inicial;
      std::set<int> estados_finales;
      std::vector<std::vector<bool>> equivalencia;
      
      bool secheck(int p, int q);
      bool check(int p, int q);
};

\end{lstlisting}

\section{Pregunta 1}
\subsection{Introducción}
\subsection{Pseudocódigo}
\subsection{Código}
\subsection{Experimentación numérica}

\section{Pregunta 2}
\subsection{Introducción}
\subsection{Pseudocódigo}
\subsection{Código}
\subsection{Experimentación numérica}


\section{Pregunta 3}
\subsection{Introducción}
\subsection{Pseudocódigo}
\subsection{Código}
\subsection{Experimentación numérica}

\section{Pregunta 4}
\subsection{Introducción}
\subsection{Pseudocódigo}
\subsection{Código}
\subsection{Experimentación numérica}

\section{Pregunta 5}
\subsection{Introducción}
\subsection{Pseudocódigo}
\subsection{Código}
\subsection{Experimentación numérica}

\end{document}
